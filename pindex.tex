% tufte-handout.tex - DESC
% Iago Mosqueira - JRC. 2013
%
\documentclass{tufte-handout}

% ams
\usepackage{amssymb,amsmath}

% Set up the images/graphics package
\usepackage{graphicx}
\setkeys{Gin}{width=\linewidth,totalheight=\textheight,keepaspectratio}
\graphicspath{{graphics/}}

% natbib
\usepackage{natbib}
\bibliographystyle{plainnat}

% biblatex

% booktabs
\usepackage{booktabs}

% url
\usepackage{url}

% hyperref
\usepackage{hyperref}

% units.
\usepackage{units}

% fancyvrb
\usepackage{fancyvrb}
\fvset{fontsize=\normalsize}
\DefineShortVerb[commandchars=\\\{\}]{\|}
\DefineVerbatimEnvironment{Highlighting}{Verbatim}{commandchars=\\\{\}}


% multiplecol
\usepackage{multicol}

% lipsum
\usepackage{lipsum}

% These commands are used to pretty-print LaTeX commands
\newcommand{\doccmd}[1]{\texttt{\textbackslash#1}}% command name -- adds backslash automatically
\newcommand{\docopt}[1]{\ensuremath{\langle}\textrm{\textit{#1}}\ensuremath{\rangle}}% optional command argument
\newcommand{\docarg}[1]{\textrm{\textit{#1}}}% (required) command argument
\newenvironment{docspec}{\begin{quote}\noindent}{\end{quote}}% command specification environment
\newcommand{\docenv}[1]{\textsf{#1}}% environment name
\newcommand{\docpkg}[1]{\texttt{#1}}% package name
\newcommand{\doccls}[1]{\texttt{#1}}% document class name
\newcommand{\docclsopt}[1]{\texttt{#1}}% document class option name

% Shaded
\newenvironment{Shaded}{}{}
\newcommand{\KeywordTok}[1]{\textcolor[rgb]{0.00,0.44,0.13}{\textbf{{#1}}}}
\newcommand{\DataTypeTok}[1]{\textcolor[rgb]{0.56,0.13,0.00}{{#1}}}
\newcommand{\DecValTok}[1]{\textcolor[rgb]{0.25,0.63,0.44}{{#1}}}
\newcommand{\BaseNTok}[1]{\textcolor[rgb]{0.25,0.63,0.44}{{#1}}}
\newcommand{\FloatTok}[1]{\textcolor[rgb]{0.25,0.63,0.44}{{#1}}}
\newcommand{\CharTok}[1]{\textcolor[rgb]{0.25,0.44,0.63}{{#1}}}
\newcommand{\StringTok}[1]{\textcolor[rgb]{0.25,0.44,0.63}{{#1}}}
\newcommand{\CommentTok}[1]{\textcolor[rgb]{0.38,0.63,0.69}{\textit{{#1}}}}
\newcommand{\OtherTok}[1]{\textcolor[rgb]{0.00,0.44,0.13}{{#1}}}
\newcommand{\AlertTok}[1]{\textcolor[rgb]{1.00,0.00,0.00}{\textbf{{#1}}}}
\newcommand{\FunctionTok}[1]{\textcolor[rgb]{0.02,0.16,0.49}{{#1}}}
\newcommand{\RegionMarkerTok}[1]{{#1}}
\newcommand{\ErrorTok}[1]{\textcolor[rgb]{1.00,0.00,0.00}{\textbf{{#1}}}}
\newcommand{\NormalTok}[1]{{#1}}

\title{Tufte Handouts in rmarkdown}
\author{Michael Sachs}

\begin{document}
\maketitle

\section{Introduction}\label{introduction}

This is an R Markdown document. Markdown is a simple formatting syntax
for authoring HTML, PDF, and MS Word documents. This package provides
output formats to create ``Tufte-style'' handouts. \footnote{For more
  details on using R Markdown see \url{http://rmarkdown.rstudio.com}.}

Tufte-style handouts make heavy use of the right margin. Our package
provides templates for creating these types of documents in pdf or html
format. The pdf format uses the tufte-handout document class.\footnote{Credit:
  \url{http://code.google.com/p/tufte-latex/}}

The html format uses bootstrap with some css to put stuff in the margin.
Each uses knitr hooks to specify the types of figures. There is also an
html version of this document. \footnote{\url{http://sachsmc.github.io/tufterhandout}}

\section{Usage}\label{usage}

To create sidenotes in html, some raw html is required. We make use of
the \texttt{aside} tag. Place the sidenote content between the tags
\texttt{\textless{}aside\textgreater{}\textless{}/aside\textgreater{}}
to place them in the sidebar. In the pdf version, simply use the pandoc
footnote format \texttt{\^{}{[}Content{]}}. Anything can be enclosed in
the \texttt{aside} tag, even code! In latex, you are limited to inline
code in the sidenotes. For instance, I can evaluate \texttt{rnorm(1)}.
\footnote{-0.5574}

The package provides two custom hooks for figure placement. The first is
\texttt{marginfigure}. Set \texttt{marginfigure = TRUE} in a chuck
option to place a figure in the right margin. Optionally, specify the
figure size and include a caption. Captions are passed through the
\texttt{fig.cap} chunk option.

\begin{Shaded}
\begin{Highlighting}[]
\KeywordTok{library}\NormalTok{(ggplot2)}
\KeywordTok{ggplot}\NormalTok{(mtcars, }\KeywordTok{aes}\NormalTok{(}\DataTypeTok{y =} \NormalTok{mpg, }\DataTypeTok{x =} \NormalTok{wt)) +}\StringTok{ }\KeywordTok{geom_point}\NormalTok{() +}\StringTok{ }
\StringTok{    }\KeywordTok{stat_smooth}\NormalTok{(}\DataTypeTok{method =} \StringTok{"lm"}\NormalTok{)}
\end{Highlighting}
\end{Shaded}

\begin{marginfigure}
 \includegraphics{./pindex_files/figure-latex/fig1}
\caption{ This is a marginfigure }
\end{marginfigure}

The html documents have the body set at a fixed width of 960px. Feel
free to edit the css to suit your needs. Html output supports any of the
built in Bootstrap themes. Be careful using the fluid grid system, it
may break the output for narrow screens.

The second custom hook is \texttt{fig.star}. Setting
\texttt{fig.star = TRUE} creates a full-width figure spanning the main
body and the margin. The caption goes in the sidebar under the figure.
These look pretty sweet! Specify the width and height for best results.

\begin{Shaded}
\begin{Highlighting}[]
\KeywordTok{ggplot}\NormalTok{(faithful, }\KeywordTok{aes}\NormalTok{(}\DataTypeTok{y =} \NormalTok{eruptions, }\DataTypeTok{x =} \NormalTok{waiting)) +}\StringTok{ }
\StringTok{    }\KeywordTok{geom_point}\NormalTok{() +}\StringTok{ }\KeywordTok{stat_smooth}\NormalTok{(}\DataTypeTok{method =} \StringTok{"loess"}\NormalTok{)}
\end{Highlighting}
\end{Shaded}

\begin{figure*}
 \includegraphics{./pindex_files/figure-latex/fig2}
\caption{ Full-width figure }
\end{figure*}

Finally, normal figures are plotted in the main body, with the captions
in the margin. The only option necessary here is the caption itself.

\begin{Shaded}
\begin{Highlighting}[]
\KeywordTok{ggplot}\NormalTok{(faithful, }\KeywordTok{aes}\NormalTok{(}\DataTypeTok{x =} \NormalTok{eruptions)) +}\StringTok{ }\KeywordTok{geom_histogram}\NormalTok{(}\DataTypeTok{binwidth =} \FloatTok{0.1}\NormalTok{)}
\end{Highlighting}
\end{Shaded}

\begin{figure}
 \includegraphics{./pindex_files/figure-latex/fig3}
\caption{ Normal figure with caption in the margin }
\end{figure}

\section{Resources}\label{resources}

Learn more about rmarkdown: \url{http://rmarkdown.rstudio.com}

\end{document}
